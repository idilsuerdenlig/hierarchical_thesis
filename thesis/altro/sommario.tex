\newpage
\chapter*{Sommario}

\addcontentsline{toc}{chapter}{Sommario}

Lo SLAM è uno dei principali problemi nello sviluppo di robot autonomi. Gli approcci correnti sono afflitti da un pesante costo computazionale e si comportano male in ambienti dinamici e disordinati; le performance di questi sistemi diventano ancora peggiori se si utilizzano sensori a basso costo, necessari per le applicazioni commerciali.
Questa tesi affronta il problema da un punto di vista nuovo e originale, usando feature ad alto livello come punti chiave e sfruttando la conoscenza di un esperto e un linguaggio fuzzy per riconoscerli, per tenere traccia di punti chiave forti e stabili e permettere mappe più intelligenti e una localizzazione robusta in ambienti complessi. L'idea fondamentale è quella di mantenere il tasso di errore della localizzazione limitato e ridurre il costo necessario a far navigare con successo un robot autonomo in un ambiente interno, usando solo i dati provenienti da una webcam e una unità di misura inerziale a basso costo.
Il principale problema è il riconoscimento di feature ad alto livello, come porte e scaffali, affrontato tramite un classificatore fuzzy ad albero, definito da un esperto, in modo da evitare fasi di allenamento e migliorare la generalità del riconoscimento.

\chapter*{Abstract}

\addcontentsline{toc}{chapter}{Abstract}

SLAM is one of the key issues in autonomous robots development. Current approaches are affected by heavy computational load and misbehave in cluttered and dynamic environment; their performance get even worse with low cost sensors, needed for market applications.
This thesis faces the problem in a new and original way, working with high level features as key points and using expert knowledge and a fuzzy language to detect them, in order to track strong and stable key points and allow smarter maps and robust localization in complex environments. The key idea is to keep the error rate of the localization process limited, and to reduce the cost of an autonomous robot to successfully navigate into an indoor environment using only the data from a webcam and a low cost Inertial measurement unit.
The main issue is the high level feature recognition, like doors and shelves, done by an expert-defined fuzzy tree classifier, in order to avoid training and improve generalization of recognition.