\chapter*{Ringraziamenti}

\addcontentsline{toc}{chapter}{Ringraziamenti}

Dopo cinque lunghi anni, sono finalmente giunto al momento di scrivere i ringraziamenti. Questo significa che sono sopravvissuto al ``Vietnam'' che è il Politecnico! Un'esperienza entusiasmante e tremendamente difficile.

Non voglio utilizzare questa pagina per fare i soliti ringraziamenti scontati: il professore mi ha aiutato, la famiglia mi ha sostenuto... Beh, penso che queste cose, seppure vere, non siano i motivi fondamentali per cui valga la pena di spendere delle parole.

Piuttosto, voglio ringraziare il Professor Bonarini per avermi dato una sfida difficile da affrontare, una sfida che mettesse alla prova le mie capacità fino al limite e che mi ha permesso di imparare così tante cose che, se me lo avessero detto prima, non ci avrei creduto. Lo ringrazio perchè devo a lui, e ai suoi collaboratori (come il buon Martino!), tutte le cose che ho imparato e che fanno la differenza tra uno studente qualunque e un vero ingegnere.

Voglio ringraziare anche Davide Cucci, per il suo fondamentale aiuto nella parte di localizzazione, quando ormai credevo di doverla escludere dalla tesi (E comunque il telefono non ha suonato!).

Voglio ringraziare anche Andrea Romanoni per essere riuscito a rispondere alle mie domande anche sotto l'ombrellone!

Voglio ringraziare i ``Nerd della prima fila'' per essere riusciti a:
\begin{itemize}[noitemsep, nolistsep]
 \item Farsi sgridare dal proprio relatore.
 \item Farsi bersagliare da gessi.
 \item Farsi sgridare da persone a caso per il casino in aula studio.
 \item Fare battute sconvenienti.
 \item Riprodurre musica e filmati sconvenienti (sempre nelle aule studio, ovvio!).
 \item Aprire siti sconvenienti da usare come esempi di applicazioni.
 \item Farsi prendere in giro perfino dai prof per la ``nerditudine''.
\end{itemize}

Voglio sottolineare che io non c'entro ASSOLUTAMENTE niente con tutte queste malefatte! Ok\dots forse 2\dots 3\dots volte\dots

Voglio infine ringraziare la mia famiglia per avermi sempre amato. Forse avrei potuto arrivare qui dove sono ora e essere quello che sono ora senza il supporto economico, gli sforzi per permetterci di seguire le lezioni e tutto il necessario ad affrontare una università impegnativa come il Politecnico. Ma sicuramente non sarei arrivato da nessuna parte senza il loro affetto. Perchè la cosa che più importa in una famiglia è sapere che la tua famiglia ci sarà sempre e comunque per te, qualsiasi cosa accada, proprio perchè si è una famiglia. Vi voglio bene.

Grazie.
