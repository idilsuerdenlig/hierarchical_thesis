\newpage
\chapter*{Abstract}

\addcontentsline{toc}{chapter}{Abstract}
Hierarchical reinforcement learning (HRL) algorithms are gaining attention as a method to enhance learning performance in terms of speed and scalability. A key idea to introduce hierarchy in reinforcement learning is to use temporal abstractions. The use of hierarchical schemes in reinforcement learning simplifies the task of designing a policy and adding prior knowledge. In addition, well-designed temporally extended actions can speed up the learning process by constraining the exploration. Furthermore, policies learned are often simpler to interpret and with fewer parameters, without losing representation power. These advantages make HRL suitable for complex robotics tasks with continuous action-spaces and high state-action dimensionality. The aim of this thesis is to adapt the approach of control theory to hierarchical schemes in the field of hierarchical reinforcement learning. We designed a novel framework to fill the gap between control theory and machine learning. 


\newpage
\chapter*{Sommario}

\addcontentsline{toc}{chapter}{Sommario}
Gli algoritmi di apprendimento per rinforzo gerarchico (Hierarchical reinforcement learning, HRL) stanno guadagnando attenzione come metodo per migliorare le prestazioni di apprendimento in termini di velocità e scalabilità. Un'idea chiave per introdurre la gerarchia nell'apprendimento per rinforzo è usare le astrazioni temporali. L'uso di schemi gerarchici nell'apprendimento per rinforzo semplifica il compito di progettare una politica e sfruttare la conoscenza di dominio. Le azioni estese temporalmente, se ben progettate, possono accelerare il processo di apprendimento, limitando l'esplorazione. Inoltre, le politiche apprese sono spesso più semplici da interpretare e con meno parametri, senza perdere potere espressivo. Questi vantaggi rendono HRL adatto a compiti di robotica complessi con spazi di azione continui e alta dimensionalità dello spazio di stato e di azione. Lo scopo di questa tesi è di adattare l'approccio agli schemi gerarchici della della teoria del controllo, nel campo dell'apprendimento per rinforzo gerarchico. Abbiamo progettato un formalismo per colmare il divario tra teoria del controllo e apprendimento automatico.